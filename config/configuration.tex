%%%%%%%%%%%%%%%%%%%%%%%%%%%%%%%%%%%%%%%%
% Settings used by \usepackage{hyperref}
%%%%%%%%%%%%%%%%%%%%%%%%%%%%%%%%%%%%%%%%
\hypersetup{
    colorlinks=true,
    citecolor=black, 
    filecolor=black, 
    linkcolor=black, 
    urlcolor=black,
}
\urlstyle{sf}

%%%%%%%%%%%%%%%%%%%%%%%%%%%%%%%%%%%%%%%%
% Settings used by \usepackage{setspace}
%%%%%%%%%%%%%%%%%%%%%%%%%%%%%%%%%%%%%%%%
% These commands are used to change the line-spacing of the text.
% See documentation at https://tex.stackexchange.com/questions/65849/
% \doublespacing    % for double-spacing
% \linespread{1.25} % for custom line spacing.
\onehalfspacing   % for 1.5 spacing


%%%%%%%%%%%%%%%%%%%%%%%%%%%%%%%%%%%%%%%%
% Settings used by \usepackage{titling}
%%%%%%%%%%%%%%%%%%%%%%%%%%%%%%%%%%%%%%%%
% The following command is used to "hang" the default LaTeX title from the top
% of the page (e.g. like a painting frame). The default LaTeX title sits a bit
% too low, so the following command sets it higher.
\setlength{\droptitle}{-2em}

%%%%%%%%%%%%%%%%%%%%%%%%%%%%%%%%%%%%%%%%
% Settings used by \usepackage{microtype}
%%%%%%%%%%%%%%%%%%%%%%%%%%%%%%%%%%%%%%%%
% These custom hyphenation rules are used to discourage excessive hyphenation
\microtypesetup{
  protrusion = true,
  expansion  = true,
  tracking   = true,
  factor     = 1000,
  patch      = all,
  final
}

%%%%%%%%%%%%%%%%%%%%%%%%%%%%%%%%%%%%%%%%
% Custom protrusion rules to allow hanging punctuation
%%%%%%%%%%%%%%%%%%%%%%%%%%%%%%%%%%%%%%%%
\SetProtrusion
{ encoding = *}
{
% char   right left
  {.} = {    , 1000},
  {,} = {    , 1000},
  {«} = {1000,     },
  {»} = {    , 1000},
  {(} = { 0  ,     },
  {)} = {    , 0   },
  {-} = {    , 500 },
  % Double Quotes
  \textquotedblleft
      = {1000,     },
  \textquotedblright
      = {    , 1000},
  \quotedblbase
      = {1000,     },
  % Single Quotes
  \textquoteleft
      = {1000,     },
  \textquoteright
      = {    , 1000},
  \quotesinglbase
      = {1000,     }
}

%%%%%%%%%%%%%%%%%%%%%%%%%%%%%%%%%%%%%%%%
% Settings used for csquotes
%%%%%%%%%%%%%%%%%%%%%%%%%%%%%%%%%%%%%%%%
\MakeOuterQuote{"}

%%%%%%%%%%%%%%%%%%%%%%%%%%%%%%%%%%%%%%%%
% Settings used by titlesec
%%%%%%%%%%%%%%%%%%%%%%%%%%%%%%%%%%%%%%%%
% For section, subsection, and subsubsection numberings to be in margin.
% Sourced from: https://tex.stackexchange.com/a/523014
\newcommand{\marginsecnumber}[1]{%
  \makebox[0pt][r]{#1\hspace{6pt}}%
}
\titleformat{\section}
  {\normalfont\Large\bfseries}
  {\marginsecnumber\thesection}
  {0pt}
  {}
\titleformat{\subsection}
  {\normalfont\large\bfseries}
  {\marginsecnumber\thesubsection}
  {0pt}
  {}
\titleformat{\subsubsection}
  {\normalfont\normalsize\bfseries}
  {\marginsecnumber\thesubsubsection}
  {0pt}
  {}
\titleformat{\paragraph}[runin]
  {\normalfont\normalsize\bfseries}
  {\marginsecnumber\theparagraph}
  {0pt}
  {}
\titleformat{\subparagraph}[runin]
  {\normalfont\normalsize\bfseries}
  {\marginsecnumber\thesubparagraph}
  {0pt}
  {}
\titlespacing*{\subsection}{0pt}{*3.25}{*1.5}%